\documentclass{beamer}
\usepackage[croatian]{babel}
\usepackage[utf8]{inputenc}
\usepackage{xcolor}
\usetheme{Ilmenau}
\definecolor{denim}{rgb}{0.08, 0.38, 0.74}
\usecolortheme[named=denim]{structure}
\begin{document}
\author[Maričević,Prpić,Miculinić]{Mario Maričević | Rea Prpić | Valentina Miculinić}
\title{*Izrada prezentacija i postera u Latexu*}
\subtitle{Seminar iz kolegija Računalne vještine}
\institute{Tehnički fakultet,Rijeka}
\date{24.1.2017.}
\frame{\titlepage}
\begin{frame}{Sadržaj}
\tableofcontents
\end{frame}

\section{Kako izraditi prezentaciju?}
\begin{frame}{Izrada prezentacije}
\begin{itemize}
	\item Neki od paketa koje možemo koristiti pri izradi su:
\end{itemize}
\begin{enumerate}
	\item \textbf{Beamer} 
	\item \textbf{Powerdot} 
	\item \textbf{Prosper... itd.}
\end{enumerate}
\begin{itemize}
	\item Svaki od njih posjeduje svoje prednosti i nedostatke.
\end{itemize}
\end{frame}
\end{document}
